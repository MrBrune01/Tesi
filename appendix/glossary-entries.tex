% Acronyms
\newacronym[description={\glslink{API}{Application Program Interface}}]
    {API}{API}{Application Program Interface}

\newacronym[description={\glslink{umlg}{Unified Modeling Language}}]
    {uml}{UML}{Unified Modeling Language}

\newacronym[description={\glslink{PoG}{Proof of Concept}}]
    {PoC}{PoC}{Proof of Concept}

\newacronym[description={\glslink{VOD}{Video On Demand}}]
    {VOD}{VOD}{Video On Demand}

    \newacronym[description={\glslink{HLS}{HTTP Live Streaming}}]
    {HLS}{HLS}{HTTP Live Streaming}
    \newacronym[description={\glslink{DASH}{Dynamic Adaptive Streaming over HTTP}}]
    {DASH}{DASH}{Dynamic Adaptive Streaming over HTTP}
    \newacronym[description={\glslink{RTMP}{Real Time Messaging Protocol}}]
    {RTMP}{RTMP}{Real Time Messaging Protocol}
    \newacronym[description={\glslink{ORM}{Object Relational Mapping}}]
    {ORM}{ORM}{Object Relational Mapping}
    \newacronym[description={\glslink{DOM}{Document Object Model}}]
    {DOM}{DOM}{Document Object Model}
    \newacronym[description={\glslink{MUI}{Material User Interface}}]
    {MUI}{MUI}{Material User Interface}
    \newacronym[description={\glslink{TUS}{TUS Upload Protocol}}]
    {TUS}{TUS}{TUS Upload Protocol}
    \newacronym[description={\glslink{IDE}{Integrated Development Environment}}]
    {IDE}{IDE}{Integrated Development Environment}
    \newacronym[description={\glslink{AAA}{Arrange Act Assert}}]
    {AAA}{AAA}{Arrange Act Assert}
    \newacronym[description={\glslink{HTTP}{HyperText Transfer Protocol}}]
    {HTTP}{HTTP}{HyperText Transfer Protocol}
    \newacronym[description={\glslink{MPEG}{Moving Picture Experts Group}}]
    {MPEG}{MPEG}{Moving Picture Experts Group}
    \newacronym[description={\glslink{TCP}{Transmission Control Protocol}}]
    {TCP}{TCP}{Transmission Control Protocol}
    \newacronym[description={\glslink{UDP}{User Datagram Protocol}}]
    {UDP}{UDP}{User Datagram Protocol}
    \newacronym[description={\glslink{DRM}{Digital Rights Management}}]
    {DRM}{DRM}{Digital Rights Management}
    \newacronym[description={\glslink{CDN}{Content Delivery Network}}]
    {CDN}{CDN}{Content Delivery Network}
    \newacronym[description={\glslink{NLB}{Network Load Balancer}}]
    {NLB}{NLB}{Network Load Balancer}
    \newacronym[description={\glslink{ALB}{Application Load Balancer}}]
    {ALB}{ALB}{Application Load Balancer}
    \newacronym[description={\glslink{RESTful}{Representational State Transfer}}]
    {RESTful}{RESTful}{Representational State Transfer}
    \newacronym[description={\glslink{LINQ}{Language Integrated Query}}]
    {LINQ}{LINQ}{Language Integrated Query}
    \newacronym[description={\glslink{UI}{User interface}}]
    {UI}{UI}{User interface}
    \newacronym[description={\glslink{TS}{Transport Stream}}]
    {TS}{TS}{Transport Stream}
    \newacronym[description={\glslink{HTML5}{HyperText Markup Language 5}}]
    {HTML5}{HTML5}{HyperText Markup Language 5}
    \newacronym[description={\glslink{DTO}{Data Transfer Object}}]
    {DTO}{DTO}{Data Transfer Object}
    \newacronym[description={\glslink{CRUD}{Create Read Update Delete}}]
    {CRUD}{CRUD}{Create Read Update Delete}
    \newacronym[description={\glslink{JSON}{JavaScript Object Notation}}]
    {JSON}{JSON}{JavaScript Object Notation}

% Glossary entries
% \newglossaryentry{apig} {
%     name=\glslink{api}{API},
%     text=Application Program Interface,
%     sort=api,
%     description={in informatica con il termine \emph{Application Programming Interface API} (ing. interfaccia di programmazione di un'applicazione) si indica ogni insieme di procedure disponibili al programmatore, di solito raggruppate a formare un set di strumenti specifici per l'espletamento di un determinato compito all'interno di un certo programma. La finalità è ottenere un'astrazione, di solito tra l'hardware e il programmatore o tra software a basso e quello ad alto livello semplificando così il lavoro di programmazione}
% }

\newglossaryentry{umlg} {
    name=\glslink{uml}{UML},
    text=UML,
    sort=uml,
    description={in ingegneria del software \emph{UML, Unified Modeling Language} (ing. linguaggio di modellazione unificato) è un linguaggio di modellazione e specifica basato sul paradigma object-oriented. L'\emph{UML} svolge un'importantissima funzione di ``lingua franca'' nella comunità della progettazione e programmazione a oggetti. Gran parte della letteratura di settore usa tale linguaggio per descrivere soluzioni analitiche e progettuali in modo sintetico e comprensibile a un vasto pubblico}
}
\newglossaryentry{WebApp} {
    name=\glslink{WebApp}{Web Application},
    text=WebApp,
    sort=WebApp,
    description={
        Una \emph{Web Application} (ing. applicazione web) è un'applicazione fruibile via web, ovvero un programma accessibile via web attraverso un network, tipicamente internet, sfruttando un'architettura tipica client-server. Le \emph{Web Application} sono applicazioni distribuite che funzionano su più dispositivi o ambienti, e sono spesso scritte in linguaggi di programmazione interpretati come \emph{PHP} o \emph{Python}. Le \emph{Web Application} sono molto popolari a causa della diffusione dei browser web e della possibilità di utilizzare un \emph{server web} per distribuire un'applicazione a più utenti
    }
}
\newglossaryentry{PoG} {
    name=\glslink{PoC}{Proof of Concept},
    text=PoC,
    sort=PoC,
    description={
        \emph{Proof of Concept} (ing. prova di fattibilità) è un termine usato in ingegneria del software e in sviluppo di sistemi per indicare un esempio o una realizzazione incompleta di un determinato metodo o idea per dimostrarne la fattibilità, in particolare per dimostrare la sua validità pratica. Un \emph{PoC} è spesso usato come dimostrazione iniziale che il progetto è fattibile e funzionante. Il \emph{PoC} è anche usato per raccogliere feedback che possono aiutare a migliorare il progetto
        }
    }

