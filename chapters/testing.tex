\chapter{Testing e validazione}
\label{cap:testing}
\intro{In questo capitolo verranno descritte le attività di testing e validazione svolte sul PoC.}\\
\section{Descrizione delle attività di testing}
Per quanto riguarda le attività di testing, sono state richieste dall'azienda solo quelle riguardanti al modulo del backend.\\
Le attività di testing sono state svolte seguendo le best practice fornite da Microsoft, che prevedono la nomenclatura dei test secondo il seguente schema:\\
NomeMetodo\_Scenario\_RisultatoAtteso in questo modo è possibile capire subito, cosa si sta testando e quale risultato ci si aspetta senza dover leggere il codice del test.\\
Inoltre, per ogni metodo testato, è stata utilizzata la metodologia \gls{AAA}, che prevede la suddivisione del test in tre parti:
\begin{itemize}
    \item \textbf{Arrange}: in questa fase si prepara lo stato iniziale necessario per eseguire il test. Può includere la creazione di oggetti, l'inizializzazione delle variabili, l'impostazione di condizioni specifiche, l'istanziazione delle dipendenze o l'impostazione di un ambiente controllato. L'obiettivo dell'Arrange è creare l'ambiente di test coerente e riproducibile.
    \item \textbf{Act}: in questa fase si esegue l'azione o il comportamento che si desidera testare. Può essere l'invocazione di un metodo, l'interazione con un'interfaccia utente, l'invio di richieste a un server, ecc. L'obiettivo dell'Act è stimolare il sistema in modo che si verifichi l'evento che si vuole testare.
    \item \textbf{Assert}: in questa fase si verifica se il comportamento del sistema durante l'azione corrisponde a quello previsto. Si stabiliscono delle asserzioni, cioè si specificano le condizioni che devono essere verificate per considerare il test superato. Se le asserzioni hanno successo, il test viene considerato valido, altrimenti viene segnalato un errore. L'obiettivo dell'Assert è controllare che l'output o lo stato del sistema sia conforme alle aspettative.
\end{itemize}

\subsection{Unit testing}
I test di unità, sono stati svolti utilizzando il framework NUnit, che permette di eseguire test di unità per applicazioni .NET.\\
Per eseguire i test di unità, è stato necessario creare un progetto di test, che permettesse di testare le funzionalità del progetto di backend. Per fare ciò, è stato necessario aggiungere il progetto di backend come riferimento al progetto di test. In questo modo, è stato possibile testare le funzionalità del progetto di backend, utilizzando i metodi pubblici delle classi del progetto di backend.\\
Il progetto di test è diviso in tre cartelle, una per ogni layer del progetto di backend. Ogni cartella contiene i test delle classi del layer corrispondente.\\
\subsubsection{Provider}
Per effettuare i test sul layer dei provider, è stato necessario creare un mock del database utilizzato dal progetto di backend. Per fare ciò, Entity Framework mette a disposizione un package NuGet chiamato \textit{EntityFrameworkCore.InMemory}, che permette di creare un database in memoria, ovvero viene creato un database temporaneo in memoria, che viene popolato con dei dati fittizi che vengono utilizzati per i test. In questo modo, è possibile testare le funzionalità del layer dei provider, senza dover utilizzare un database reale. Inoltre viene passato al costruttore del provider, un oggetto di tipo \textit{DbContextOptions}, che permette di testare i metodi del provider, utilizzando il database in memoria.\\

\subsubsection{Service}
Per effettuare i test sul layer Core, è stato necessario anche qui creare un mock del database come nel layer dei provider, successivamente è stata istanziata la classe del Provider che si vuole testare, passando al costruttore un oggetto di tipo \textit{DbContextOptions}, che permette di utilizzare i metodi del provider, utilizzando il database in memoria. In seguito è stata istanziata la classe del Service utilizzando come costruttori i provider precedentemente creati.\\

\subsubsection{Controller}
I test sul layer dei controller non sono stati svolti in quanto non potendo effettuare il mock dei servizi esterni, non è stato possibile instanziare le classi Services necessarie per testare i controller.\\

\section{Analisi dei risultati}
Le attività di testing sono state svolte in maniera parziale in quanto sia per mancanza di tempo, non è stato possibile testare tutte le funzionalità del backend, in particolare i controller e i metodi dei Service che utilizzano i servizi esterni. Sicuramente in futuro, sarà necessario effettuare dei test anche su queste funzionalità, in modo da garantire un prodotto di qualità.\\ 
