\chapter{Fondamenti teorici}
\label{cap:fondamentiteorici}
\intro{In questo capitolo verranno descritti i fondamenti teorici necessari per la comprensione del lavoro svolto.}\\

\section{Concetti di video on demand e streaming}
Video on demand (tradotto come video su richiesta) o VOD, è un sistema che permette di accedere a contenuti multimediali(video, audio, immagini) 
in qualsiasi momento e in qualsiasi luogo tramite una connessione internet.
Contrariamente alla trasmissione televisiva tradizionale, nella quale gli utenti sono limitati da un palinsesto prefedinito, 
il VOD dà la possibilità agli utenti di scegliere quale contenuto guardare e quando usufruirne.\\
Nel contesto della webapp sviluppata, questo concetto ha una funzione chiave, infatti consente agli utenti di accedere a una vasta gamma di selezione video riguardanti 
i prodotti esposti nelle fiere mondilali, 
permettendo di scegliere i video di loro interesse in base alle loro preferenze e neccessità, eliminando le limitazioni spazio-temporali delle fiere fisiche.\\
Nel corso della tesi, verranno analizzate le caratteristiche e le sfide associate all'implementazione del video on demand nella webapp, comprese le strategie 
di gestione e organizzazione dei contenuti, nonché la scalabilità e la qualità dello streaming per garantire un'esperienza fluida e coinvolgente per gli utenti.
\subsection{Tipologie di video on demand}
Esistono diverse tipologie di video on demand, sotto elencate:

\begin{itemize}
    \item \textbf{Subscription VOD} ovvero i servizi con un canone periodico come ad esempio Netflix, Amazon Prime Video ecc..\\
    \item \textbf{Transactional VOD} ossia servizi che permettono di acquistare o noleggiare contenuti, come ad esempio Google Play, Apple TV, Chili ecc..\\
    \item \textbf{Advertising VOD} ossia servizi gratuiti che mostrano annunci pubblicitari durante la riproduzione dei contenuti, come ad esempio Youtube, RaiPlay e Mediaset Play\\
    \item \textbf{Premium VOD} ovvero la trasmissione di contenuti Premium, come antemprime cinematografiche, eventi sportivi ecc.., proposti da piattaforme come ad esempio Curzon Cinemas\\
\end{itemize}

\subsection{Concetti di streming video}
Lo streaming video è un metodo di trasmissione di dati multimediali, in particolare di video e audio.
Esistono due principali categorie di streaming video: live streaming e video on demand.\\
\begin{itemize}
    \item \textbf{Video on demand} è la trasmissione di contenuti pre-registrati, come ad esempio film, serie TV, documentari ecc.., i quali vengono compressi e memorizzati su un server come file,
    e vengono trasmessi agli utenti che ne fanno richiesta senza la necessità che il contenuto venga scaricato sul dispositivo dell'utente. Infatti i dati ricevuti dalla richiesta vengono decompressi e riprodotti in tempo reale.
    \item \textbf{Live streaming} è simile alle trasmissioni televisive tradizionali, in cui gli utenti guardano i contenuti in tempo reale. Viene utilizzato per trasmettere eventi in 
    diretta come ad esempio concerti, eventi sportivi ecc.., vengono anch'essi leggermente compressi e memorizzati su un server, ma vengono trasmessi in tempo reale agli utenti che ne fanno richiesta.
    \end{itemize}
\subsection{Protocolli e tecnologie di streaming video}
Per la trasmissione di contenuti multimediali, esistono diversi protocolli e tecnologie, sotto elencati:

\begin{itemize}

    \item \textbf{HTTP Live Streaming} o HLS è un protocollo di streaming sviluppato da Apple nel 2009, permette la trasmissione in streaming di contenuti multimediali,
    divide il contenuto in file HTTP più piccoli e scaricabili, chiamati segmenti e li distribuisce ai dispositivi client tramite HTTP.\\
    HLS è un protocollo di streaming adattivo, ovvero il client può cambiare la qualità del video in base alla larghezza di banda disponibile senza interrompere la riproduzione.
    È nativamente compatibile con i dispotivi Apple ed è supportato dalla maggior parte dei dispositivi e browser che supportano HTTP, non richiedendo l'installazione di plugin aggiuntivi.\\

    \item \textbf{Dynamic Adaptive Streaming over HTTP} o DASH è un protocollo di streaming sviluppato dal Moving Picture Experts Group (MPEG), permette la trasmissione di contenuti multimediali
    attraverso il protocollo HTTP.\\
    DASH suddivide il contenuto in segmenti e li trasmette ai dispositivi client tramite HTTP, permettendo un adattamento dinamico della qualità del video in base alla larghezza di banda disponibile.
    Offre una vasta gamma di scekta del formato video e codec, permettendo di scegliere il formato più adatto per il dispositivo client.\\

    \item \textbf{Real Time Messaging Protocol} o RTMP è un protocollo di streaming sviluppato da Adobe nel 2012, permette la trasmissione di contenuti multimediali in tempo reale,
    divide il contenuto in pacchetti e li trasmette ai dispositivi client tramite TCP o UDP, consente una comunicazione bidirezionale tra il server e il dispositivo client utilizzando un flusso continuo.\\
    RTMP è un protocollo di streaming non adattivo, ovvero non permette di cambiare la qualità del video in base alla larghezza di banda disponibile, ma permette di 
    trasmettere contenuti in tempo reale con una bassa latenza.\\
    Dal 2020, con la deprecazione di Adobe Flash Player, RTMP è stato sostituito da protocolli di streaming adattivi come HLS e DASH.\\



    \end{itemize}


\section{Tecnlogie utilizzate}

\section{Archittetura di una webapp}

\section{Funzionalita della webapp}