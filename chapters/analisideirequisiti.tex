\chapter{Analisi dei requisiti}
\label{cap:analisi-requisiti}
\intro{In questo capitolo verranno descritti i requisiti individuati per lo sviluppo del POC.}\\

\section{Individuazione e specifica dei requisiti funzionali e non funzionali}

I requisiti individuati sono stati classificati in base alla loro tipologia, funzionali o non funzionali, e in base alla loro priorità, obbligatori, desiderabili o opzionali.\\
I requisiti funzionali sono stati individuati in base alle funzionalità che il sistema deve offrire, mentre i requisiti non funzionali sono stati individuati in base alle caratteristiche che il sistema deve avere.\\
I requisiti obbligatori sono quelli che il sistema deve assolutamente soddisfare, i requisiti desiderabili sono quelli che il sistema dovrebbe soddisfare, mentre i requisiti opzionali sono quelli che il sistema potrebbe soddisfare.\\

\subsection{Requisiti funzionali}

\begin{itemize}
    \item \textbf{RF1}: l'utente deve poter creare un nuovo evento;
    \item \textbf{RF2}: l'utente deve poter modificare un evento esistente;
    \item \textbf{RF3}: l'utente deve poter eliminare un evento esistente;
    \item \textbf{RF4}: l'utente deve poter visualizzare la tabella degli eventi;
    \item \textbf{RF5}: l'utente deve poter caricare un video e associarlo ad un evento;
    \item \textbf{RF6}: l'utente deve poter visualizzare la tabella dei video;
    \item \textbf{RF7}: l'utente deve poter modificare un video esistente;
    \item \textbf{RF8}: l'utente deve poter eliminare un video esistente;
    \item \textbf{RF9}: l'utente deve poter visualizzare la lista dei video associati ad un evento;
    \item \textbf{RF10}: l'utente deve poter aggiungere un utente al sistema;
    \item \textbf{RF11}: l'utente deve poter modificare un utente esistente;
    \item \textbf{RF12}: l'utente deve poter eliminare un utente esistente;
\end{itemize}

\subsection{Requisiti non funzionali}

% \begin{itemize}
%     % \item 
%     \end{itemize}


\section{Casi d'uso}
