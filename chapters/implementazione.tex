\chapter{Implementazione}
\label{cap:implementazione}

L'implementazione del PoC è una parte fondamentale del progetto di stage, in quanto permette di verificare la fattibilità del prodotto e di valutare la bontà delle scelte progettuali.\\
In questo capitolo verranno descritte le principali scelte implementative, sia per quanto riguarda il frontend che il backend.\\

\section{Struttura del progetto}
La struttura del progetto è stata organizzata in modo da separare il frontend dal backend, in modo da poterli sviluppare in modo indipendente. Inoltre, è stato scelto di utilizzare una repository su Azure DevOps per il versioning del codice, in modo da poter avere uno storicizzazione delle modifiche.
\section{Backend}
Il backend è stato sviluppato utilizzando il linguaggio di programmazione C\texttt{\#} e il framework ASP.NET Core. 
Come ambiente di sviluppo è stato utilizzato Visual Studio 2022.\\
Il backend è stato sviluppato seguendo il template fornito dall'azienda, che prevede l'utilizzo di un'architettura a layer.\\
\subsection{Layer}
L'architettura a layer è una tipologia di architettura software che prevede la suddivisione del codice in diversi livelli, ognuno dei quali ha un compito ben preciso, e che comunica con gli altri livelli solo attraverso interfacce. Questo permette di avere un codice più modulare e manutenibile, in quanto ogni livello ha un compito ben preciso e non si occupa di altro.\\
Nel backend sono stati implementati i seguenti layer: API, Core e Data.\\
\subsubsection{API}
Il layer API è il livello più esterno dell'applicazione, e si occupa di gestire le richieste HTTP in arrivo dal frontend. Per fare ciò, utilizza il framework ASP.NET Core, che permette di gestire le richieste HTTP in modo semplice e veloce.
È divisa in due parti: Controllers e DTOs.\\
% \paragraph{Controllers}
I Controllers sono le classi che si occupano di gestire le richieste HTTP in arrivo dal frontend.\\
ASP.NET Core permette di gestire le richieste HTTP in modo semplice e veloce, grazie all'utilizzo dei Controllers.\\
I Controllers sono classi che estendono la classe \texttt{ControllerBase}, e che contengono i metodi che gestiscono le richieste HTTP.\\
Per ogni entità del database è stato creato un controller, che contiene i metodi per gestire le richieste HTTP.\\
% \paragraph{DTOs}
I DTOs (Data Transfer Object) sono classi che contengono i dati che vengono scambiati tra il frontend e il backend.\\
ASP.NET Core permette di gestire le richieste HTTP in modo semplice e veloce, grazie all'utilizzo dei Controllers.\\
I Controllers sono classi che estendono la classe \texttt{ControllerBase}, e che contengono i metodi che gestiscono le richieste HTTP.\\
Per ogni entità del database è stato creato un controller, che contiene i metodi per gestire le richieste HTTP.\\


\section{Frontend}


\section{Integrazione con Azure Media Services}