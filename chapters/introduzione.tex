\chapter{Introduzione}
\label{cap:introduzione}

% \noindent Esempio di citazione nel pie' di pagina \\
% citazione\footcite{womak:lean-thinking} \\
\section{L'idea}
Nell'attuale panorama delle fiere e degli eventi commerciali, le aziende partecipanti hanno manifestato un crescente interesse nella promozione innovativa
dei propri prodotti. Attualmente, essa avviene principalmente attraverso la distribuzione di materiale pubblicitario, come brochure, volantini e cataloghi, 
lasciando al visitatore il compito di informarsi autonomamente sui prodotti offerti dai vari espositori.\\
In questa tesi verrà descritto lo sviluppo di una \gls{WebApp} attraverso la quale gli espositori avranno la possibilità di caricare i propri video
relativi ai prodotti esposti, rendendoli successivamente disponibili per la riproduzione di \gls{VOD} da parte dei visitatori.
L'idea rappresenta un passo avanti verso l'innovazione della promozione dei prodotti nelle fiere, perché offre ai partecipanti un'esperienza interattiva 
e coinvolgente.\\

\section{Descrizione dello stage}

L'azienda ha manifestato l'esigenza di sviluppare un \gls{PoG} per la realizzazione di una WebApp permetta agli espositori di caricare i propri video relativi ai prodotti esposti, e li renda disponibili per la riproduzione on-demand da parte dei visitatori.\\
L'idea è quella di realizzare un prodotto che possa essere utilizzato in occasione di fiere ed eventi commerciali.\\
L'applicazione è stata sviluppata utilizzando come linguaggio di backend C\texttt{\#}~\footnote{\url{https://learn.microsoft.com/it-it/dotnet/csharp/}}, per lo sviluppo del frontend il framework JavaScript React~\footnote{\url{https://reactjs.org/}}  ,
mentre per la gestione dell'archiviazione e streaming dei video sono stati utilizzati i servizi di Microsoft Azure: Media Service, Account Storage, SQL Server e App Service.~\footnote{\url{https://azure.microsoft.com/}}\\
L'obbiettivo principiale di questo PoC, è stato quello di studiare e verificare la fattibilità di un prodotto di questo tipo.\\
\section{L'azienda}

Ad Maiora Studio è una software house nata nel 2013 per operare nel campo del mobile e che negli anni si è specializzata anche nello sviluppo di software.~\footnote{\url{https://admaiorastudio.com/}}\\
La sua mission è centrata sull'attenzione verso i clienti e lo sviluppo di software moderni, scalabili e progettati ad hoc per soddisfare le loro esigenze. \\
I prodotti sviluppati da Ad Maiora Studio si basano sulle più recenti tecnologie, integrano componenti e librerie eterogenee e pongono una forte enfasi sull'esperienza utente
 e l'interfaccia grafica.\\
L'azienda si distingue per l'approccio continuativo di assistenza ai clienti, 
garantendo un partner sempre accessibile e in grado di rispondere tempestivamente alle richieste. \\
Ad Maiora Studio si concentra principalmente su piccole e medie imprese operanti nei settori industriale e dei servizi, incoraggiandole a intraprendere un percorso di modernizzazione
 iniziando dal software.\\
L'obiettivo è supportare efficacemente l'adattamento alle mutevoli esigenze di mercato e di business, fornendo strumenti innovativi e personalizzati, che rappresentano il cuore dell'attività di Ad Maiora Studio.\\
Grazie alla competenza Full Stack del team di sviluppatori, l'azienda è in grado di realizzare ogni tipo di software, 
coprendo l'intero processo di sviluppo, dalla progettazione all'implementazione.\\
La qualità delle soluzioni software offerte è sempre un punto focale, al fine di soddisfare appieno le aspettative dei clienti e garantire il massimo risultato.



\section{Struttura della tesi}

\begin{description}

    \item[{\hyperref[cap:introduzione]{Il primo capitolo}}] introduce l'idea del progetto, la descrizione dello stage e l'azienda
    \item[{\hyperref[cap:fondamentiteorici]{Il secondo capitolo}}] descrive i fondamenti teorici alla base dello streaming video e le tecnologie utilizzate per la realizzazione del PoC
    
    \item[{\hyperref[cap:analisi-requisiti]{Il terzo capitolo}}] descrive gli obiettivi obbligatori, desiderabili e facoltativi, i prodotti attesi e la pianificazione del lavoro.
    
    \item[{\hyperref[cap:progettazione]{Il quarto capitolo}}] approfondisce la fase di progettazione dell'applicazione e come vengono implementate le tecnologie tra di loro per il raggiungimento degli obiettivi.
    
    \item[{\hyperref[cap:implementazione]{Il quinto capitolo}}] descrive il processo di implementazione andando nel dettaglio delle tecnologie utilizzate.
    
    \item[{\hyperref[cap:testing]{Il sesto capitolo}}] approfondisce la fase di Unit Test del backend e dei risultati ottenuti

    \item[{\hyperref[cap:analisi-costi]{Il settimo capitolo}}] descrive l'analisi dei costi effettuata per il mantenimento dell'applicazione in base al numero di utenti.

    
    \item[{\hyperref[cap:conclusioni]{L'ottavo capitolo}}] descrive il raggiungimento degli obiettivi, le conoscenze acquisite e la valutazione personale del progetto.
\end{description}

