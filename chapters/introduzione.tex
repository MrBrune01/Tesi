\chapter{Introduzione}
\label{cap:introduzione}

% Introduzione al contesto applicativo.\\

% \noindent Esempio di utilizzo di un termine nel glossario \\
% \gls{api}. \\

% \noindent Esempio di citazione in linea \\
% \cite{site:agile-manifesto}. \\

% \noindent Esempio di citazione nel pie' di pagina \\
% citazione\footcite{womak:lean-thinking} \\
\section{L'idea}

Nell'attuale panorama delle fiere e degli eventi commerciali, le aziende partecipanti hanno manifestato un crescente interesse nell'innovazione della promozione 
dei propri prodotti. Attualmente, la promozione avviene principalmente attraverso la distribuzione di materiale pubblicitario, come brochure, volantini e cataloghi, 
lasciando al visitatore il compito di informarsi autonomamente sui prodotti offerti dai vari espositori.\\
Con l'obiettivo di rendere l'esperienza del visitatore più semplificata e innovativa, durante il mio tirocinio presso l'azienda AdMaioraStudio ho avuto 
l'opportunità di sviluppare un prototipo di web app. Questa web app consente agli espositori di promuovere i propri prodotti in maniera innovativa, 
offrendo ai visitatori la possibilità di fruire video on demand relativi ai prodotti esposti da vari espositori in diverse fiere mondiali.\\
L'obiettivo principale di questa esperienza è stato la realizzazione di un Proof of Concept (POC) al fine di studiare e verificare la fattibilità di 
un prodotto di questo tipo. Attraverso l'app sviluppata, gli espositori hanno la possibilità di caricare i propri video relativi ai prodotti esposti, 
rendendoli successivamente disponibili per la visualizzazione on demand da parte dei visitatori.\\
Questo progetto rappresenta un passo avanti verso l'innovazione nella promozione dei prodotti nelle fiere, offrendo ai visitatori un'esperienza interattiva 
e coinvolgente. Durante lo sviluppo del prototipo, ho avuto l'opportunità di applicare le competenze acquisite nel corso di studi, 
approfondendo la programmazione in \lstinline[language={[Sharp]C}]|C#| per il backend e utilizzando tecnologie come Azure Media Service e React per il frontend.\\
Attraverso questo lavoro di stage, si è mirato non solo a creare una soluzione concreta per i visitatori delle fiere, ma anche a gettare le basi per ulteriori 
sviluppi e miglioramenti futuri nel campo della promozione innovativa dei prodotti attraverso l'utilizzo dei video on demand.\\


\section{L'azienda}

AdMaioraStudio è un'azienda di consulenza informatica che si occupa di sviluppo software, consulenza e formazione.\\
L'azienda è nata nel 2013 inizialmente nell'ambito mobile, successivamente ha ampliato il proporio target transitando nello sviluppo software a tutto tondo.\\
AdMaioraStudio è attualmente composta da 7 sviluppatori, ognuno del quale ha competenze di sviluppo Full Stack.\\
Il target di clienti alla quale l'azienda si rivolge sono le piccole e medie aziende nel ramo industriale e dei servizi.\\

\section{Obbietti della tesi}
//TODO
//obbiettivi della tesi\\


\section{Strutta della tesi}

\begin{description}
    \item[{\hyperref[cap:processi-metodologie]{Il secondo capitolo}}] descrive i fondamenti teorici
    
    \item[{\hyperref[cap:descrizione-stage]{Il terzo capitolo}}] descrive l'analisi dei requisiti
    
    \item[{\hyperref[cap:analisi-requisiti]{Il quarto capitolo}}] approfondisce la progettazione
    
    \item[{\hyperref[cap:progettazione-codifica]{Il quinto capitolo}}] approfondisce l'implementazione
    
    \item[{\hyperref[cap:verifica-validazione]{Il sesto capitolo}}] approfondisce la fase di testing e validazione
    
    \item[{\hyperref[cap:conclusioni]{Nel settimo capitolo}}] descrive le conclusioni e i possibili sviluppi futuri
\end{description}

Riguardo la stesura del testo, relativamente al documento sono state adottate le seguenti convenzioni tipografiche:
\begin{itemize}
	\item gli acronimi, le abbreviazioni e i termini ambigui o di uso non comune menzionati vengono definiti nel glossario, situato alla fine del presente documento;
	\item per la prima occorrenza dei termini riportati nel glossario viene utilizzata la seguente nomenclatura: \emph{parola}\glsfirstoccur;
	\item i termini in lingua straniera o facenti parti del gergo tecnico sono evidenziati con il carattere \emph{corsivo}.
\end{itemize}
