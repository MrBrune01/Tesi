\chapter{Progettazione}
\label{cap:progettazione}

La progettazione del sistema è un passo fondamentale per lo sviluppo di un sistema software.\\
In questo capitolo verranno descritte le scelte progettuali effettuate per la progettazione del sistema, in particolare per lo sviluppo del database, il frontend, il backend e l'integrazione con Azure Media Services.\\
\section{Progettazione del database}
La progettazione del database è stata effettuata sulla base dei requisiti funzionali e non funzionali individuati in fase di analisi.\\
Il database è stato progettato per memorizzare i dati relativi agli eventi, video e utenti.\\
Per la sua progettazione è stato utilizzato il framework Entity Framework Core, che permette di definire il modello del database utilizzando classi e proprietà direttamente dal codice. Non è stato necessario scrivere codice SQL per la creazione del database, ma è stato sufficiente definire le classi e le proprietà dei modelli e successivamente eseguire le migrazioni per aggiornare il database.\\

\clearpage

\subsection{Modello del database}
Il database è composto da tre tabelle: Event, Video e User.\\
La tabella Event contiene i dati relativi agli eventi, come l'id, il titolo, l'id dell'user che l'ha creato, la data di inizio e di fine e l'URL alla thumbanil; la tabella Video contiene i dati relativi ai video, come l'id, il titolo, l'URL alla thumbnail, l'URL al video e l'id dell'evento a cui appartiene; la tabella User contiene i dati relativi agli utenti, come l'id e il nome;

\begin{figure}[H] 
    \centering 
    \includegraphics[scale=0.5]{diagrammi/database.png} 
    \caption{Diagramma del database}
\end{figure}

\section{Progettazione del frontend}
La progettazione del frontend è sviluppata sulla base dell'utilizzo di React come libreria principale per lo sviluppo dell'interfaccia utente.
%  Verranno illustrati i principi di progettazione di React, come la creazione di componenti riutilizzabili, la gestione dello stato dell'applicazione e la definizione delle rotte per la navigazione. Saranno presentati anche i diagrammi dell'architettura del frontend, mostrando come i componenti si integrano per creare un'esperienza utente coerente.
\subsection{Principi di progettazione di React}
React è una libreria JavaScript per la creazione di interfacce utente. È stata sviluppata da Facebook e viene utilizzata per la creazione di UI per applicazioni web e mobile.\\
React è basato su sei principi di progettazione: Componenti, State, DOM, Route, Context e Props.\\
\subsubsection{Componenti}
Le componenti sono elementi dell'interfaccia utente che possono essere riutilizzate in diverse parti dell'applicazione, una componente può essere una piccola porzione di pagina o un elemento complesso e autonomo, consentono di creare interfacce utente modulari e riutilizzabili, in modo tale da rendere il codice più semplice da scrivere, leggere e mantenere.\\
\subsubsection{State}
Lo stato è un oggetto JavaScript che contiene i dati che vengono utilizzati dai componenti dell'applicazione, è immutabile, quindi non può essere modificato direttamente; per modificarlo, è necessario utilizzare il metodo \texttt{setState()} che viene fornito da React; quando lo stato viene modificato, viene aggiornato automaticamente il DOM.\\
\subsubsection{DOM}
Il DOM (Document Object Model) è una rappresentazione virtuale degli elementi della pagina, quando avvengono cambiamenti nello stato dell'applicazione, React aggiorna automaticamente il DOM in modo efficiente e successivamente aggiorna solo le parti della pagina che sono state modificate, così facendo React rende l'applicazione più veloce e reattiva senza dover ricaricare l'intera pagina.\\
\subsubsection{Route}
Le route vengono utilizzate per gestire la navigazione e la visualizzazione delle diverse pagine dell'applicazione, consentono di definire le corrispondenze tra gli url specifici e i componenti che devono essere visualizzati quando viene richiesto un url specifico.
Per gestire il routing, React utilizza una libreria esterna chiamata React Router, la quale fornisce diverse componenti che consentono di definire le route e di stabilire le corrispondenze tra gli url e i componenti.\\
\subsubsection{Context}
Il Context è un meccanismo che consente di condividere dati specifici con tutti i componenti figli di un componente padre, evitando di dover passare manualmente le props attraverso i livelli intermedi, è composto da due parti: il Provider e il Consumer; il primo è responsabile di definire il contesto e di fornire i dati, mentre il secondo accede ai dati forniti dal Provider.\\
\subsubsection{Props}
Le props (abbreviazione di proprietà) sono oggetti JavaScript immutabili che vengono utilizzati per passare dati da un componente padre a un componente figlio in modo unidirezionale.
Il passaggio di dati tra la componente padre e la componente figlio avviene tramite gli attributi di quest'ultimo, mentre il passaggio di dati tra la componente figlio e la componente padre avviene tramite le funzioni callback.\\

\subsection{Architettura del frontend}
Il frontend è sviluppato utilizzando un template disposto dall'azienda, che utilizza il design pattern Container-Presenter, è composto da due componenti principali: il Container e il Presenter.
Il primo è responsabile della gestione dello stato dell'applicazione, dell'interazione con i dati e della logica di business, si occupa di recuperare i dati, gestire gli eventi, effettuare chiamate API e gestire lo stato globale dell'applicazione; il secondo invece, è responsabile dell'aspetto visuale e dell'interfaccia utente, riceve i dati e le funzioni dai Container e si occupa di renderizzare l'interfaccia utente in base ai dati ricevuti.\\
La comunicazione tra i due avviene tramite le props, in quanto il Container passa i dati al Presenter tramite props, mentre il Presenter invia le informazioni tramite callback fornite dal Container.\\

\subsection{Diagrammi dell'architettura del frontend}
\subsubsection{Diagramma dell'architettura}
\begin{figure}[!h] 
    \centering 
    \includegraphics[scale=0.5]{diagrammi/frontend.drawio.png} 
    \caption{Diagramma design pattern Container-Presenter}
\end{figure}
\clearpage
\section{Progettazione del backend}
La progettazione del backend è sviluppata sulla base dell'utilizzo del linguaggio di programmazione C\texttt{\#} e del framework ASP.NET Core.\\

\subsection{Architettura del backend}
Il backend è sviluppato utilizzando un template disposto dall'azienda, che utilizza una separazione delle componenti in layer distinti, dove ognuno ha un compito specifico.\\
Il template è composto da tre layer fondamentali: API, Core e Data; il primo è responsabile della comunicazione con il frontend, il secondo è responsabile della gestione dello stato dell'applicazione e il terzo è responsabile della comunicazione con il database.\\

\subsubsection{API}
È responsabile della comunicazione con il frontend, è composto da due parti: i controller e i DTO. Il primo ha il compito di gestire le richieste HTTP provenienti dal frontend e coordinare le azioni richieste per soddisfarle: quando un controller riceve una richiesta, estrae i dati necessari dalla richiesta e interagisce con i layer Core per fornire una risposta al Client. 
Il secondo ha il compito di definire la struttura dei dati che vengono trasferiti tra il frontend e il backend durante la chiamata, in modo da consentire una comunicazione standardizzata e senza ambiguità. I DTO possono includere solo i campi necessari per soddisfare una certa richiesta, così facendo, riducono il trasferimento di dati inutili e rendono la comunicazione più veloce; oltre a ciò, sono utilizzati anche per la validazione dei dati, garantendo che i dati ricevuti dal frontend siano validi.\\
\subsubsection{Core}
Il layer Core è responsabile della gestione dello stato dell'applicazione. Riceve le richieste dal layer API e le elabora, interagendo con il layer Data per ottenere i dati necessari; è composto da due parti: i models e i service. I models rappresentano la struttura dei dati che vengono utilizzati dall'applicazione per effettuare le operazioni richieste; i service, invece, gestiscono la logica dell'applicazione, contengono i metodi che vengono chiamati dai controller, che eseguono operazioni con i servizi esterni e con il layer Data.\\

\subsubsection{Data}
Il layer Data è responsabile dell'accesso ai dati, è composto da tre parti: Context, Entity e Provider. Il primo ha il compito di gestire la connessione con il database, definisce la struttura del database e fornisce i metodi per accedervi; il secondo ha il compito di definire la struttura dei dati che vengono salvati nel database; il terzo ha il compito di gestire la comunicazione con il database, fornisce i metodi per accedere ai dati per effettuare le operazioni di lettura e scrittura.\\

\subsection{Diagrammi dell'architettura del backend}
\subsubsection{Diagramma dell'architettura}
\begin{figure}[H] 
    \centering 
    \includegraphics[scale=0.5]{diagrammi/backend.png} 
    \caption{Diagramma architettura del backend}
\end{figure}

\section{Integrazione con Azure}
Azure è una piattaforma cloud proprietaria di Microsoft che offre vari servizi.
La WebApp è integrata con Azure per la gestione del database, gestione dei video e per la distribuzione dell'applicazione, sono utilizzati i seguenti servizi: SQL Server, Azure Media Service e Azure App Service.\\

\subsection{Azure SQL Server}
Azure SQL Server è un servizio di database relazionale completamente gestito che offre funzionalità di database SQL.
Offre una serie di vantaggi rispetto a un Server SQL tradizionale, come la facilità di gestione, la scalabilità e sicurezza.\\
Il database non deve essere gestito in quanto Azure si occupa della gestione dell'infrastruttura, aggiornamenti delle patch di sicurezza e delle operazioni di manutenzione del server, consentendo agli sviluppatori di concentrarsi sulla logica di business e sull'applicazione dati.\\
Inoltre permette di scalare sia orizzontalmente che verticalmente il database, in maniera automatica, in modo tale da adattarsi alle esigenze dell'applicazione.\\
Un altro vantaggio è la sicurezza, in quanto integra funzionalità di sicurezza avanzate, come la crittografia dei dati in transito e a riposo, la protezione da minacce e la gestione degli accessi.\\
È stato utilizzato per la gestione dei dati dell'applicazione, integrandosi con il backend dell'applicazione attraverso il context contenuto nel layer Data.\\

\subsection{Azure Media Service}
Azure Media Service è un servizio che permette la gestione e la distribuzione di contenuti multimediali su diverse piattaforme e dispositivi.\\
Offre varie funzionalità, ma per gli scopi di questa applicazione è stato utilizzato per la codifica, archiviazione dei video e la distribuzione dei video.\\

\subsubsection{Codifica}
La codifica è un processo che permette di convertire un video in un formato compatibile con la maggior parte dei dispositivi e delle piattaforme.\\
Azure Media Service permette di codificare i video in diversi formati e risoluzioni, in modo da rendere disponibile la visione del video su qualsiasi dispositivo e in qualsiasi condizione di rete.\\
Per lo sviluppo di questo PoC è stato deciso di codificare i video in H.264 e utilizzando il preset Adaptive Streaming, che permette di codificare il video in diversi formati e risoluzioni, in modo da adattarsi alla qualità della rete e al dispositivo utilizzato. Sono state scelte queste impostazioni che permettono di ottenere un video di risoluzione massima pari a 1080p 

\subsubsection{Archiviazione}
Una volta codificato, il video viene salvato in Azure Storage Account, un servizio di archiviazione di dati non strutturati come file, immagini e video.\\
Ogni video viene salvato in un container, che è una cartella che contiene i video codificati in varie risoluzioni, il manifest e la thumbanil generata automaticamente da Azure Media Service.\\

\subsubsection{Distribuzione}
Una volta archiviato, il video viene distribuito attraverso degli Streaming Endpoint, che consentono di creare dei punti di accesso per lo streaming di contenuti multimediali in diretta o on-demand, offre la possibilità di distribuire i video attraverso vari protocolli di streaming; per questo PoC è stato utilizzato il protocollo HLS.\\
L'utilizzo di Streaming Endpoint offre vari vantaggi, il più importante è la scalabilità, in quanto permette di scalare automaticamente il numero di istanze in base al numero di richieste, in modo da garantire fluidità e continuità dello streaming senza interruzioni.\\
La distribuzione avviene attraverso un URL, generato da Azure Media Service, che punta al manifest del video, che viene salvato nel database dell'applicazione insieme agli altri dati del video.\\


\subsection{Azure App Service}
È un servizio di Azure che permette di distribuire e gestire applicazioni web e API senza dover gestire l'infrastruttura.\\
Il depoly dell'applicazione avviene attraverso l'interfaccia di Visual Studio 2022, che permette di pubblicare l'applicazione direttamente su Azure App Service.\\
Permette di scalare automaticamente il numero di istanze in base al numero di richieste, in modo da garantire fluidità e continuità dell'applicazione senza interruzzioni.\\
Inoltre permette di gestire il dominio dell'applicazione, in modo da poter utilizzare un dominio personalizzato.\\

In generale, l'integrazione con Azure permette di ridurre i costi di gestione e di manutenzione dell'infrastruttura, in quanto è tutto gestito in automatico, permettendo di concentrarsi sullo sviluppo dell'applicazione più che sulla gestione dell'infrastruttura.\\