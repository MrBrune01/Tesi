\chapter{Progettazione}
\label{cap:progettazione}

La progettazione del sistema è un passo fondamentale per lo sviluppo di un sistema software.\\
In questo capitolo verranno descritte le scelte progettuali effettuate per la progettazione del sistema, in particolare verranno descritte le scelte progettuali effettuate per lo sviluppo del database, il frontend, il backend e l'integrazione con Azure Media Services.\\

% \section{Descrizione dell'architettura del sistema}

% Inizieremo con una panoramica dell'architettura generale del sistema. Saranno illustrate le diverse componenti del sistema, 
% come il frontend, il backend e i servizi di Azure utilizzati per lo streaming e lo storage dei video. 
% Verranno presentati i diagrammi che mostrano come queste componenti interagiscono tra loro per fornire la funzionalità desiderata.

\section{Progettazione del database}

\section{Progettazione del frontend}
La progettazione del frontend è sviluppata sulla base dell'utilizzo di React come libreria principale per lo sviluppo dell'interfaccia utente.
%  Verranno illustrati i principi di progettazione di React, come la creazione di componenti riutilizzabili, la gestione dello stato dell'applicazione e la definizione delle rotte per la navigazione. Saranno presentati anche i diagrammi dell'architettura del frontend, mostrando come i componenti si integrano per creare un'esperienza utente coerente.
\subsection{Principi di progettazione di React}
React è una libreria JavaScript per la creazione di interfacce utente. È stata sviluppata da Facebook e viene utilizzata per la creazione di UI per applicazioni web e mobile.\\
React è basato su sei principi di progettazione: componenti, state, DOM, route, context e props.\\
\subsubsection{Componenti}
Le componenti sono elementi dell'interfaccia utente che possono essere riutilizzate in diverse parti dell'applicazione.\\
Una componente può essere una piccola porzione di pagina o un'elemento complesso e autonomo, consentono di creare interfacce utente modulari e riutilizzabili, in modo tale da rendere il codice più semplice da scrivere, leggere e mantenere.\\
\subsubsection{State}
Lo stato è un oggetto JavaScript che contiene i dati che vengono utilizzati dai componenti dell'applicazione.\\
Lo stato è immutabile, quindi non può essere modificato direttamente, per modificarlo, è necessario utilizzare il metodo \texttt{setState()} che viene fornito da React.\\
Quando lo stato viene modificato, viene aggiornato automaticamente il DOM.\\
\subsubsection{DOM}
Il DOM (Document Object Model) è una rappresentazione virtuale degli elementi della pagina.\\
Quando avvengono cambiamenti nello stato dell'applicazione, React aggiorna automaticamente il DOM in modo efficiente e successivamente aggiorna solo le parti della pagina che sono state modificate.\\
In questo modo React rende l'applicazione più veloce e reattiva senza dover ricaricare l'intera pagina.\\
\subsubsection{Route}
Le route vengono utilizzate per gestire la navigazione e la visualizzazione delle diverse pagine dell'applicazione.\\
Consento di definire le corrispondenze tra gli URL specifici e i componenti che devono essere visualizzati quando viene richiesto un URL specifico.\\
Per gestire il routing, React utilizza una libreria esterna chiamata React Router, la quale fornisce diverse componenti che consentono di definire le route e di stabilire le corrispondenze tra gli URL e i componenti.\\
\subsubsection{Context}
Il Context è un meccanismo che consente di condividere dati specifici con tutti i componenti figli di un componente padre, evitando di dover passare manualmente le props attraverso i livelli intermedi.\\
È composto da due parti: il Consumer e il Provider.\\
Il Consumer è il componente che utilizza i dati condivisi, mentre il Provider è il componente che condivide i dati.\\
\subsubsection{Props}
Le props (abbreviazione di proprietà) sono oggetti JavaScript immutabili che vengono utilizzati per passare dati da un componente padre a un componente figlio in modo unidirezionale.\\
Il passaggio di dati tra la componente padre e la componente figlio avviene tramite gli attributi di quest'ultimo.\\

\subsection{Architettura del frontend}
Il frontend è sviluppato utilizzando un template disposto dall'azienda, che utilizza il design pattern Container-Presenter, ovvero una variante del design pattern Model-View-Controller con la differenza che il Controller è sostituito dal Presenter.\\
Il design pattern Container-Presenter è composto da due componenti principali: il Container e il Presenter, il primo è responsabile della gestione dello stato dell'applicazione e della comunicazione con il backend, mentre il secondo il responsabile della presentazione dei dati e della gestione degli eventi.\\
La comunicazione tra i due avviene tramite le props, dove il Container passa le props al Presenter, che le utilizza per presentare i dati e per gestire gli eventi, mentre il Presenter può comunicare con il Container tramite le props, ad esempio può chiamare una funzione del Container passata come prop.\\
\subsection{Diagrammi dell'architettura del frontend}
\subsubsection{Diagramma dei componenti}

% % \begin{figure}[H]
% %     \centering
% %     \includegraphics[width=1\textwidth]{images/frontend-components-diagram.png}
% %     \caption{Diagramma dei componenti del frontend}
% %     \label{fig:frontend-components-diagram}
% % \end{figure}
TODO immagine diagramma dei componenti

\section{Progettazione del backend}
La progettazione del backend è sviluppata sulla base dell'utilizzo di .NET Core come framework principale per lo sviluppo del backend.\\


\subsection{Architettura del backend}
Il backend è sviluppato utilizzando un template disposto dall'azienda, che utilizza una separazione delle componenti in layer distinti, dove ognuno ha un compito specifico.\\
Il template è composto da tre layer fondamentali: API, Core e Data.\\
API è responsabile della comunicazione con il frontend.\\
Core è responsabile della gestione dello stato dell'applicazione.\\
Data è responsabile della comunicazione con il database.\\

\subsubsection{API}
Il layer API è responsabile della comunicazione con il frontend, è composto da due parti: i controller e i DTO.\\
Il primo ha il compito di gestire le richieste HTTP provenienti dal frontend e coordinare le azioni richieste per soddisfarle.\\
Quando un controller riceve una richiesta HTTP, estrae i dati necessari dalla richiesta e interagisce con i layer Core per fornire una risposta al Client.\\
Il secondo ha il compito di definire la struttura dei dati che vengono trasferiti tra il frontend e il backend, in modo da consentire una comunicazione standardizzata e senza ambiguità.\\
I DTO possono includere solo i campi necessari per soddisfare una certa richiesta, così facendo, riduce il trasferimento di dati inutili e rende la comunicazione più veloce.\\
Oltre a ciò, sono utilizzati anche per la validazione dei dati, garantendo che i dati ricevuti dal frontend siano validi.\\
\subsubsection{Core}
Il layer Core è responsabile della gestione dello stato dell'applicazione. Riceve le richieste dal layer API e le elabora, interagendo con il layer Data per ottenere i dati necessari.\\
È composto da due parti: i models e i service.\\
I models rappresentano la struttura dei dati che vengono utilizzati dall'applicazione per effettuare le operazioni richieste.\\
I service, invece gestiscono la logica dell'applicazione, contengono i metodi che vengono chiamati dai controller, che eseguono operazioni con i servizi esterni e con il layer Data.\\

\subsubsection{Data}
Il layer Data è responsabile dell'accesso ai dati, è composto da tre parti: Context, Entity e Provider.\\
Il primo ha il compito di gestire la connessione con il database, definisce la struttura del database e fornisce i metodi per accedervi.\\
Il secondo ha il compito di definire la struttura dei dati che vengono salvati nel database.\\
Il terzo ha il compito di gestire la comunicazione con il database, fornisce i metodi per accedere ai dati per effettuare le operazioni di lettura e scrittura.\\

\subsection{Diagrammi dell'architettura del backend}
\subsubsection{Diagramma dei componenti}

TODO immagine diagramma dei componenti

\section{Integrazione con Azure}
Azure è una piattaforma cloud proprietaria di Microsoft che offre vari servizi.
La WebApp è integrata con Azure per la gestione del database, gestione dei video e per la distribuzione dell'applicazione.\\
Per lo sviluppo di questa applicazione sono utilizzati i seguenti servizi: SQL Server, Azure Media Service e Azure App Service.\\

\subsection{Azure SQL Server}
Azure SQL Server è un servizio di database relazionale completamente gestito che offre funzionalità di database SQL.
Offre una serie di vantaggi rispetto a un Server SQL tradizionale, come la facilità di gestione, la scalabilità e sicurezza \
Il database non deve essere gestito in quanto Azure si occupa della gestione dell'infrastruttura, aggiornamenti delle patch di sicurezza e delle operazioni di manutenzione del server, consentendo agli sviluppatori di concentrarsi sulla logica di business e sull'applicazioni dati.\\
Inoltre permette di scalare sia orizzontalmente che verticalmente il database, in maniera automatica, in modo tale da adattarsi alle esigenze dell'applicazione.\\
Un altro vantaggio è la sicurezza, in quanto integra funzionalità di sicurezza avanzate, come la crittografia dei dati in transito e a riposo, la protezione da minacce e la gestione degli accessi.\\
È stato utilizzato per la gestione dei dati dell'applicazione, integrandosi con il backend dell'applicazione attraverso il context contenuto nel layer Data.\\

\subsection{Azure Media Service}
Azure Media Service è un servizio di streaming video e audio che offre funzionalità di streaming video e audio, è stato utilizzato per la gestione dei video dell'applicazione.\\


\subsection{Azure App Service}
Azure App Service è un servizio di hosting web completamente gestito che offre funzionalità di hosting web, è stato utilizzato per la distribuzione dell'applicazione.\\
