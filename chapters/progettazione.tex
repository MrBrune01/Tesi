\chapter{Progettazione}
\label{cap:progettazione}

La progettazione del sistema è un passo fondamentale per lo sviluppo di un sistema software.\\
In questo capitolo verranno descritte le scelte progettuali effettuate per lo sviluppo del sistema.\\
In particolare verranno descritte le scelte progettuali effettuate per il database, per il frontend, per il backend e per l'integrazione con Azure Media Services.\\

\section{Descrizione dell'architettura del sistema}

Inizieremo con una panoramica dell'architettura generale del sistema. Saranno illustrate le diverse componenti del sistema, 
come il frontend, il backend e i servizi di Azure utilizzati per lo streaming e lo storage dei video. 
Verranno presentati i diagrammi che mostrano come queste componenti interagiscono tra loro per fornire la funzionalità desiderata.

\section{Progettazione del database}

\section{Progettazione del frontend in React}

Nella progettazione del frontend, ci concentreremo sull'utilizzo di React come libreria principale per lo sviluppo dell'interfaccia utente. Saranno illustrati i principi di progettazione di React, come la creazione di componenti riutilizzabili, la gestione dello stato dell'applicazione e la definizione delle rotte per la navigazione. Saranno presentati anche i diagrammi dell'architettura del frontend, mostrando come i componenti si integrano per creare un'esperienza utente coerente.

\subsection{Principi di progettazione di React}

React è una libreria JavaScript per la creazione di interfacce utente. È stata sviluppata da Facebook e viene utilizzata per la creazione di UI per applicazioni web e mobile.\\
React è basato su quattro principi di progettazione: componenti, stato, rotte e proprietà.\\

\begin{itemize}
    \item{Componenti}: sono elementi dell'interfaccia utente che possono essere riutilizzati in diverse parti dell'applicazione. Una componente può essere una piccola porzione di pagina o un'elemento complesso e autonomo.\\
    I componenti consentono di creare interfacce utente modulari e riutilizzabili, in modo tale da rendere il codice più semplice da scrivere, leggere e mantenere.\\
    \item{State}: è un oggetto JavaScript che contiene i dati che vengono utilizzati dai componenti dell'applicazione. Lo stato è immutabile, quindi non può essere modificato direttamente. Per modificare lo stato, è necessario utilizzare il metodo \texttt{setState()} che viene fornito da React, quando lo stato viene modificato, viene aggiornato automaticamente il DOM.\\
    \item{DOM}: è un albero di oggetti che rappresenta la struttura di un documento HTML. Il DOM è una rappresentazione del documento HTML che può essere modificata utilizzando JavaScript.\\
    \item{Props}: vengono utilizzate per passare dati da un componente padre ad un componente figlio, ad esempio un componente padre può passare una funzione ad un componente figlio, in questo modo il componente figlio può chiamare la funzione del componente padre.\\
    \item{Route}:
    \item{Context}:
\end{itemize}

\subsection{Architettura del frontend}

Il frontend è sviluppato utilizzando un template disposto dall'azienda, che utilizza il design pattern Container-Presenter, ovvero una variante del design pattern Model-View-Controller che la differenza che il Controller è sostituito dal Presenter.\\
Il design pattern Container-Presenter è composto da due componenti principali: il Container e il Presenter.\\
Il Container è il responsabile della gestione dello stato dell'applicazione e della comunicazione con il backend, mentre il Presenter è il responsabile della presentazione dei dati e della gestione degli eventi.\\
La comunicazione tra il Container e il Presenter avviene tramite le props.\\
Il Container passa le props al Presenter, che le utilizza per presentare i dati e per gestire gli eventi.\\
Il Presenter può comunicare con il Container tramite le props, ad esempio può chiamare una funzione del Container passata come prop.\\





\section{Progettazione del backend in C Sharp}

\section{Integrazione con Azure Media Services}