\chapter{Conclusioni}
\label{cap:conclusioni}

\section{Raggiungimento degli obiettivi}
Il PoC sviluppato permette agli utenti la visualizzazione dei video e la possibilità di cambiare la qualità di visualizzazione, dà la possibilità agli espositori di caricare i video relativi ai propri prodotti e di gestirli e permette agli amministratori di gestire gli utenti, gli eventi e i video.\\
È stato sviluppato utilizzando le tecnologie richieste dall'azienda, in particolare React per il frontend, .NET Core per il backend e Azure per la gestione dei video e del database.\\
Inoltre sono stati sviluppati parzialmente gli unit test per il backend per verificare il corretto funzionamento delle funzionalità.\\
Successivamente è stata effettuata un analisi dei costi per il mantenimento dell'applicazione in base al numero di utenti utilizzatori dell'applicazione.\\
Infine è stata redatta la documentazione di resoconto finale, che descrive il lavoro svolto e i risultati ottenuti.\\
Concludendo, tutti gli obiettivi obbligatori sono stati raggiunti, tranne la realizzazione degli Unit Test completa del backend a causa di mancanza di tempo.\\
\section{Sviluppi futuri}
Il PoC sviluppato è una buona base per la realizzazione del prodotto finale, ma necessita di alcuni miglioramenti.\\
In particolare, l'aggiunta di un sistema di autenticazione per gli utenti, in modo da poter accedere all'applicazione solo se registrati, e di un sistema di autenticazione per gli espositori, in modo da poter caricare i video solo se autenticati.\\
Inoltre, sarebbe utile aggiungere un sistema di engagement per gli utenti, in modo da poter interagire con gli espositori e con gli altri utenti tramite commenti e valutazione dei prodotti.\\
Infine, è necessario aggiungere un sistema di controllo degli errori, in modo da poter gestire gli errori che possono verificarsi durante l'utilizzo dell'applicazione.\\

\section{Conoscenze acquisite}
Grazie a questa esperienza ho avuto l'opportunita di imparare nuove tecnologie come React, il framework .NET Core e il sistema di Azure, sicuramente molto utili per il mio futuro professionale. Inoltre, ho potuto approfondire le mie conoscenze in ambito di sviluppo web e di architetture software.\\
\section{Valutazione personale}
Sono molto soddisfatto del lavoro svolto, in particolare per la realizzazione del prototipo funzionante, che mi ha permesso di mettere in pratica le conoscenze acquisite durante il corso di studi e di imparare nuove tecnologie. Sicuramente il prodotto finale sarà molto utile per l'azienda presso la quale ho svolto tirocinio, in quanto ritengo sia una buona base dalla quale partire per la realizzazione del prodotto finale.\\ 
Questa esperienza ha confermato che il percorso di studi che ho scelto è quello giusto per me e mi ha approcciato verso il mondo del lavoro, che ho potuto conoscere meglio.\\