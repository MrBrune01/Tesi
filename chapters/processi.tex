\chapter{Descrizione dello stage}
\intro{In questo capitolo verrà descritto lo stage, partendo dalla descrizione di esso, gli obiettivi, la pianificazione, i risultati attesi ecc..}\\
\label{cap:descrizione-stage}
\section{Obiettivi}
\label{sec:obiettivi}
Seguono gli obbietti identificati nel piano di lavoro.
\subsection{Notazione}
\label{subsec:notazione}
Gli obiettivi sono cosi identificati:
\begin{itemize}
    \item \textbf{O} per gli obiettivi obbligatori;
    \item \textbf{D} per gli obiettivi desiderabili;
    \item \textbf{F} per gli obiettivi facoltativi.
\end{itemize}
\subsection{Obiettivi obbligatori}
obiettivi obbligatori:
\begin{itemize}
    \item \textbf{O1} Realizzazione del POC;
    \item \textbf{O2} Sviluppo Unit test backend;
    \item \textbf{O3} Analisi dei costi;
    \item \textbf{O4} Stesura della documentazione.
\end{itemize}
\subsection{Obiettivi desiderabili}
obiettivi desiderabili:
\begin{itemize}
    \item \textbf{D1} Analisi approfondita e comparazione di diverse architetture e sul loro impatto sui costi;
    \item \textbf{D2} Analisi dei competitor per osservare la gestione dei problemi di carico elevato.
\end{itemize}
\subsection{Obiettivi facoltativi}
obiettivi facoltativi:
\begin{itemize}
    \item \textbf{F1} Analisi dei sviluppi futuri;
    \item \textbf{F2} Realizzazione di un'interfaccia grafica, funzionale e conforme agli standard qualitativi di un'applicazione moderna.

\end{itemize}
\section{Pianificazione}
\label{sec:pianificazione}
\begin{table}[!h]
    \label{tab:ripartizione-ore}
    \begin{tabularx}{\textwidth}{|c|X|}
        \hline
        \textbf{Durata in ore} & \textbf{Descrizione dell'attività} \\\hline
        
        \textbf{80} & \textbf{Formazione sulle tecnologie} \\ \hdashline
            \multirow{2}{0cm}\\
                \textit{30} &
                \textit{Studio delle basi dello stack tecnologico utilizzato dall'azienda e dei metodi di sviluppo} \\
                \textit{50} &
                \textit{Studio e comparazione dei servizi offerti da Azure per realizzare la web app. Analisi di pro e contro dei vari servizi.} \\
        \hline
        
        \textbf{180} & \textbf{Sviluppo} \\ \hdashline 
        \multirow{3}{0cm}\\ 
        \textit{40} & 
        \textit{Realizzazione della struttura di base della web app considerando le analisi di cui sopra.} \\
        \textit{100} & 
        \textit{Implementazione di front-end, back-end e DB} \\
        \textit{80} & 
        \textit{Test, ottimizzazione, refactoring. Realizzazione di unit test.} \\
        \hline
        
        \textbf{40} & \textbf{Documentazione e Demo}  \\ \hdashline 
        \multirow{4}{0cm}\\ 
        \textit{25} & 
        \textit{Collaudo} \\
        \textit{10} & 
        \textit{Stesura documentazione finale} \\
        \textit{1} & 
        \textit{Incontro di presentazione della piattaforma con gli stakeholders} \\
        \textit{2} & 
        \textit{Live demo di tutto il lavoro di stage} \\
        \hline
        
        \textbf{Totale ore} & \multicolumn{1}{|c|}{\textbf{300}} \\\hline
        
        
    \end{tabularx}
    \caption{Tabella di ripartizione delle ore}

\end{table}
\section{Prodotti attesi}
\label{sec:prodotti-attesi}
\begin{itemize}
    \item \textbf{POC} \\
    Realizzazione di un POC funzionante che dimostri la fattibilità dell'idea proposta.
    \item \textbf{Unit test} \\
    Realizzazione completa degli unit test per il backend.\\
    \item \textbf{Analisi dei costi} \\
    Analisi dei costi simulando vari scenari di utilizzo della piattaforma.\\
    \item \textbf{Documentazione} \\
    Documentazione completa di tutte le fasi di sviluppo e analisi.
    \begin{enumerate}
        \item Lettura e documentazione \\
         Documentare le varie opzioni disponibili per lo sviluppo di un'applicazione cloud, tecniche e tecnologie utilizzate e dei loro sviluppi futuri.
        \item Sviluppo e analisi \\
        Preparazione dell'ambiente di sviluppo, sviluppo del POC e delle sue componenti.
        \item Conclusioni \\
        Documentazione completa degli artefatti sviluppati e definizione dei possibili casi d'uso.
    \end{enumerate}
\end{itemize}
\section{Risorse messe a disposizione}
\label{sec:risorse-messe-a-disposizione}
Per lo svolgimento dello stage sono state messe a disposizione le seguenti risorse:
\begin{itemize}
    \item \textbf{Account Azure DevOps} \\
    Account Azure DevOps per la gestione del progetto, delle sue attività e del versionamento del codice.
    \item \textbf{Account Azure} \\
    Account Azure per la gestione delle risorse cloud.
\end{itemize}

\section{Processo sviluppo prodotto}
Il processo di sviluppo seguirà la pianificazione descritta nella sezione \ref{sec:pianificazione} e sarà suddiviso nelle seguenti fasi:
\begin{enumerate}
    \item \textbf{Formazione} \\
    Studio delle tecnologie e dei metodi di sviluppo utilizzati dall'azienda.
    \item \textbf{Analisi} \\
    Analisi delle tecnologie e dei servizi offerti da Azure per la realizzazione del POC.
    \item \textbf{Sviluppo} \\
    Sviluppo del POC e delle sue componenti.
    \item \textbf{Collaudo} \\
    Collaudo del POC e documentazione finale.
\end{enumerate}
