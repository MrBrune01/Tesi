\chapter{Descrizione dello stage}
\label{cap:analisi-requisiti}
\intro{In questo capitolo verrà descritto lo stage, partendo dalla descrizione di esso, gli obiettivi, la pianificazione, i risultati attesi ecc..}\\
\label{cap:descrizione-stage}
\section{Obiettivi}
\label{sec:obiettivi}
L'obiettivo principale di questo progetto di stage è quello di realizzare un PoC di una WebApp che permetta il caricamento e la visualizzazione di video in streaming utilizzando i servizi di Azure, realizzando la progettazione e lo sviluppo sia del modulo backend che del modulo frontend.
\subsection{Notazione}
\label{subsec:notazione}
Gli obiettivi sono cosi identificati:
\begin{itemize}
    \item \textbf{O} per gli obiettivi obbligatori;
    \item \textbf{D} per gli obiettivi desiderabili;
    \item \textbf{F} per gli obiettivi facoltativi.
\end{itemize}
\subsection{Obiettivi obbligatori}
\label{sec:requsitiobb}
Vengono elencati gli obiettivi obbligatori:
\begin{itemize}
    \item \textbf{O1} Realizzazione del PoC;
    \begin{itemize}
        \item\textbf{O1.1}: Si deve poter aggiungere un evento;
        \item\textbf{O1.2}: Si deve poter aggiungere un utente;
        \item\textbf{O1.3}: Si deve poter aggiungere un video ad un evento;
        \item\textbf{O1.4}: Si deve poter modificare i dati di un evento;
        \item\textbf{O1.5}: Si deve poter modificare i dati relativi ad un utente;
        \item\textbf{O1.6}: Si deve poter modificare i dati relativi ad un video;
        \item\textbf{O1.7}: Si deve poter visualizzare in streaming un video precedentemente caricato;
        \item\textbf{O1.8}: Si deve poter controllare lo stato di un video;
        \item\textbf{O1.9}: Si deve poter modificare la qualità del video che viene ripodotto;
        \item\textbf{O1.10}: Si deve poter eliminare un evento;
        \item\textbf{O1.11}: Si deve poter eliminare un user;
        \item\textbf{O1.12}: Si deve poter eliminare un video.
    \end{itemize}
    \item \textbf{O2} Sviluppo Unit test backend;
    \item \textbf{O3} Analisi dei costi di mantenimento della WebApp;
    \item \textbf{O4} Stesura della documentazione relativa alla progettazione e sviluppo della WebApp.
\end{itemize}
\subsection{Obiettivi desiderabili}
Vengono elencati gli obiettivi desiderabili:
\begin{itemize}
    \item \textbf{D1} Analisi approfondita e comparazione di diverse architetture e sul loro impatto sui costi;
    \item \textbf{D2} Analisi dei competitor per osservare la gestione dei problemi di carico elevato.
\end{itemize}
\subsection{Obiettivi facoltativi}
Vengono elencati gli obiettivi facoltativi:
\begin{itemize}
    \item \textbf{F1} Analisi dei sviluppi futuri;
    \item \textbf{F2} Realizzazione di un'interfaccia grafica, funzionale e conforme agli standard qualitativi di un'applicazione moderna.

\end{itemize}
\section{Prodotti attesi}
\label{sec:prodotti-attesi}
Vengono elencati i prodotti attesi durante il periodo di stage.
\begin{itemize}
    \item \textbf{POC} \\
    Realizzazione di un PoC funzionante che rispetti i requisiti obbligatori elencati sopra \ref{sec:requsitiobb}
    \item \textbf{Unit test} \\
    Realizzazione completa degli unit test per il backend.\\
    \item \textbf{Analisi dei costi} \\
    Analisi dei costi di mantenimento simulando vari scenari di utilizzo della piattaforma.\\
    \item \textbf{Documentazione} \\
    Documentazione completa di tutte le fasi di sviluppo e analisi.
    \begin{enumerate}
        \item Lettura e documentazione \\
         Documentare le varie opzioni disponibili per lo sviluppo di un'applicazione cloud, tecniche e tecnologie utilizzate e dei loro sviluppi futuri.
        \item Sviluppo e analisi \\
        Preparazione dell'ambiente di sviluppo, sviluppo del POC e delle sue componenti.
        \item Conclusioni \\
        Documentazione completa degli artefatti sviluppati e definizione dei possibili casi d'uso.
    \end{enumerate}
\end{itemize}
\section{Risorse messe a disposizione}
\label{sec:risorse-messe-a-disposizione}
Per lo svolgimento dello stage sono state messe a disposizione le seguenti risorse:
\begin{itemize}
    \item \textbf{Account Azure DevOps} \\
    Account Azure DevOps per la gestione del progetto, delle sue attività e del versionamento del codice.
    \item \textbf{Account Azure} \\
    Account Azure per la gestione delle risorse cloud.
\end{itemize}

\section{Processo sviluppo prodotto}
Il processo di sviluppo seguirà la pianificazione descritta nel piano di lavoro e sarà suddiviso nelle seguenti fasi:
\begin{enumerate}
    \item \textbf{Formazione} \\
    Studio delle tecnologie e dei metodi di sviluppo utilizzati dall'azienda.
    \item \textbf{Analisi} \\
    Analisi delle tecnologie e dei servizi offerti da Azure per la realizzazione del POC.
    \item \textbf{Sviluppo} \\
    Sviluppo del POC e delle sue componenti.
    \item \textbf{Collaudo} \\
    Collaudo del POC e documentazione finale.
\end{enumerate}
