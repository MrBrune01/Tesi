\chapter{Descrizione dello stage}
\intro{In questo capitolo verrà descritto lo stage, paretendo dalla descrizione di esso, gli obbiettivi, la pianificazione, i risultati attesi ecc..}\\
\label{cap:descrizione-stage}
\section{Obiettivi}
\label{sec:obiettivi}
Seguono gli obbietti identificati nel piano di lavoro.

\subsection{Notazione}
\label{subsec:notazione}
Gli obbiettivi sono cosi identificati:

\begin{itemize}
    \item \textbf{O} per gli obbiettivi obbligatori;
    \item \textbf{D} per gli obbiettivi desiderabili;
    \item \textbf{F} per gli obbiettivi facoltativi.
\end{itemize}

\subsection{Obbiettivi obbligatori}
Obbiettivi obbligatori:

\begin{itemize}
    \item \textbf{O1} Realizzazione del POC;
    \item \textbf{O2} Simulazione di un utilizzo massimo dell'applicazione;
    \item \textbf{O3} Analisi dei costi;
    \item \textbf{O4} Testing del backend
    \item \textbf{O5} Stesura della documentazione;

\end{itemize}
\subsection{Obbiettivi desiderabili}
Obbiettivi desiderabili:
\begin{itemize}
    \item \textbf{D1} Analisi approfondita e comparazione di diverse architetture e sul loro impatto sui costi;
    \item \textbf{D2} Analisi dei competitor per osservare la gestione dei problemi di carico elevato;

\end{itemize}
\subsection{Obbiettivi facoltativi}
Obbiettivi facoltativi:
\begin{itemize}
    \item \textbf{F1} Analisi dei sviluppi futuri;
    \item \textbf{F2} Realizzazione di un'interfaccia grafica, funzionale e conforme agli standard qualitativi di un'applicazione moderna;

\end{itemize}

\section{Pianificazione}
\label{sec:pianificazione}
\begin{table}%
    \caption{Tabella di ripartizione delle ore}
    \label{tab:ripartizione-ore}
    \begin{tabularx}{\textwidth}{lXl}
    \hline
    \textbf{Durata in ore} & \textbf{Descrizione dell'attività}\
    \hline
    80 & Formazione sulle tecnologie \
    30 & Studio delle basi dello stack tecnologico utilizzato dall'azienda e dei metodi di sviluppo \
    50 & Studio e comparazione dei servizi offerti da Azure per realizzare la web app. Analisi di pro e contro dei vari servizi. \
    \hline
    180 & Sviluppo \
    40 & Realizzazione della struttura di base della web app considerando le analisi di cui sopra \
    100 & Implementazione di front-end, back-end e DB \
    80 & Test, ottimizzazione, refactoring. Realizzazione di unit test. \
    \hline
    40 & Documentazione e Demo\
    25 & Collaudo\
    10 & Stesura documentazione\
    1 & Incontro di presentazione della piattaforma con gli stakeholders\
    2 & Live demo di tutto il lavoro di stage\
    \hline
    \end{tabularx}
    \end{table}

\section{Prodotti attesi}
\section*{Prodotti attesi}
\label{sec:prodotti-attesi}

\begin{itemize}
    \item \textbf{POC} \\
    Realizzazione di un POC che simuli un'applicazione cloud con un carico elevato di richieste.
    \item \textbf{Documentazione} \\
    Documentazione completa di tutte le fasi di sviluppo e analisi.
    \begin{enumerate}
        \item Lettura e documentazione \\
         Documentare le varie opzioni disponibili per lo sviluppo di un'applicazione cloud, tecniche e tecnologie utilizzate e dei loro sviluppi futuri.
        
        \item Sviluppo e analisi \\
        Preparazione dell'ambiente di sviluppo, sviluppo del POC e delle sue componenti.
    
        
        \item Conclusioni \\
        Documentazione completa degli artefatti sviluppati e definizione dei possibili casi d'uso.
    \end{enumerate}
\end{itemize}



\section{Risorse messe a disposizione}
\label{sec:risorse-messe-a-disposizione}
Per lo svolgimento dello stage sono state messe a disposizione le seguenti risorse:
\begin{itemize}
    \item \textbf{Account Azure DevOps} \\
    Account Azure DevOps per la gestione del progetto, delle sue attività e del versionamento del codice.
    \item \textbf{Account Azure} \\
    Account Azure per la gestione delle risorse cloud.
\end{itemize}

\section{Processo sviluppo prodotto}
