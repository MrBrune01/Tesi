% \omiss produces '[...]'
\newcommand{\omissis}{[\dots\negthinspace]}

% Itemize symbols
% see: https://tex.stackexchange.com/a/62497
% \renewcommand{\labelitemi}{$\bullet$}
% \renewcommand{\labelitemii}{$\cdot$}
% \renewcommand{\labelitemiii}{$\diamond$}
% \renewcommand{\labelitemiv}{$\ast$}

% Custom hyphenation rules
\hyphenation {
    e-sem-pio
    ex-am-ple
}

% Images path
\graphicspath{{images/}}

% Page format settings
% see: http://wwwcdf.pd.infn.it/AppuntiLinux/a2547.htm
\setlength{\parindent}{14pt}    % first row indentation
\setlength{\parskip}{0pt}       % paragraphs spacing

% Load variables
\newcommand{\myName}{Mattia Brunello}
\newcommand{\myTitle}{Applicazione Cloud per la riproduzione video on-demand a qualità dinamica}
\newcommand{\myDegree}{Tesi di laurea}
\newcommand{\myUni}{Università degli Studi di Padova}
\newcommand{\myFaculty}{Corso di Laurea in Informatica}
\newcommand{\myDepartment}{Dipartimento di Matematica ``Tullio Levi-Civita''}
\newcommand{\profTitle}{Prof.}
\newcommand{\myProf}{Ombretta Gaggi}
\newcommand{\myLocation}{Padova}
\newcommand{\myAA}{2022-2023}
\newcommand{\myTime}{Luglio 2023}

% PDF file metadata fields
% when updating them delete the build directory, otherwise they won't change
\RequirePackage{filecontents}
\begin{filecontents*}{\jobname.xmpdata}
  \Title{Document's title}
  \Author{Mattia Brunello}
  \Language{it-IT}
  \Subject{Short description}
  \Keywords{keyword1\sep keyword2\sep keyword3}
\end{filecontents*}


% Acronyms
\newacronym[description={\glslink{API}{Application Program Interface}}]
    {API}{API}{Application Program Interface}

\newacronym[description={\glslink{umlg}{Unified Modeling Language}}]
    {uml}{UML}{Unified Modeling Language}

\newacronym[description={\glslink{PoG}{Proof of Concept}}]
    {PoC}{PoC}{Proof of Concept}

\newacronym[description={\glslink{VOD}{Video On Demand}}]
    {VOD}{VOD}{Video On Demand}

    \newacronym[description={\glslink{HLS}{HTTP Live Streaming}}]
    {HLS}{HLS}{HTTP Live Streaming}
    \newacronym[description={\glslink{DASH}{Dynamic Adaptive Streaming over HTTP}}]
    {DASH}{DASH}{Dynamic Adaptive Streaming over HTTP}
    \newacronym[description={\glslink{RTMP}{Real Time Messaging Protocol}}]
    {RTMP}{RTMP}{Real Time Messaging Protocol}
    \newacronym[description={\glslink{ORM}{Object Relational Mapping}}]
    {ORM}{ORM}{Object Relational Mapping}
    \newacronym[description={\glslink{DOM}{Document Object Model}}]
    {DOM}{DOM}{Document Object Model}
    \newacronym[description={\glslink{MUI}{Material User Interface}}]
    {MUI}{MUI}{Material User Interface}
    \newacronym[description={\glslink{TUS}{TUS Upload Protocol}}]
    {TUS}{TUS}{TUS Upload Protocol}
    \newacronym[description={\glslink{IDE}{Integrated Development Environment}}]
    {IDE}{IDE}{Integrated Development Environment}
    \newacronym[description={\glslink{AAA}{Arrange Act Assert}}]
    {AAA}{AAA}{Arrange Act Assert}
    \newacronym[description={\glslink{HTTP}{HyperText Transfer Protocol}}]
    {HTTP}{HTTP}{HyperText Transfer Protocol}
    \newacronym[description={\glslink{MPEG}{Moving Picture Experts Group}}]
    {MPEG}{MPEG}{Moving Picture Experts Group}
    \newacronym[description={\glslink{TCP}{Transmission Control Protocol}}]
    {TCP}{TCP}{Transmission Control Protocol}
    \newacronym[description={\glslink{UDP}{User Datagram Protocol}}]
    {UDP}{UDP}{User Datagram Protocol}
    \newacronym[description={\glslink{DRM}{Digital Rights Management}}]
    {DRM}{DRM}{Digital Rights Management}
    \newacronym[description={\glslink{CDN}{Content Delivery Network}}]
    {CDN}{CDN}{Content Delivery Network}
    \newacronym[description={\glslink{NLB}{Network Load Balancer}}]
    {NLB}{NLB}{Network Load Balancer}
    \newacronym[description={\glslink{ALB}{Application Load Balancer}}]
    {ALB}{ALB}{Application Load Balancer}
    \newacronym[description={\glslink{RESTful}{Representational State Transfer}}]
    {RESTful}{RESTful}{Representational State Transfer}
    \newacronym[description={\glslink{LINQ}{Language Integrated Query}}]
    {LINQ}{LINQ}{Language Integrated Query}
    \newacronym[description={\glslink{UI}{User interface}}]
    {UI}{UI}{User interface}
    \newacronym[description={\glslink{TS}{Transport Stream}}]
    {TS}{TS}{Transport Stream}
    \newacronym[description={\glslink{HTML5}{HyperText Markup Language 5}}]
    {HTML5}{HTML5}{HyperText Markup Language 5}
    \newacronym[description={\glslink{DTO}{Data Transfer Object}}]
    {DTO}{DTO}{Data Transfer Object}
    \newacronym[description={\glslink{CRUD}{Create Read Update Delete}}]
    {CRUD}{CRUD}{Create Read Update Delete}
    \newacronym[description={\glslink{JSON}{JavaScript Object Notation}}]
    {JSON}{JSON}{JavaScript Object Notation}

% Glossary entries
% \newglossaryentry{apig} {
%     name=\glslink{api}{API},
%     text=Application Program Interface,
%     sort=api,
%     description={in informatica con il termine \emph{Application Programming Interface API} (ing. interfaccia di programmazione di un'applicazione) si indica ogni insieme di procedure disponibili al programmatore, di solito raggruppate a formare un set di strumenti specifici per l'espletamento di un determinato compito all'interno di un certo programma. La finalità è ottenere un'astrazione, di solito tra l'hardware e il programmatore o tra software a basso e quello ad alto livello semplificando così il lavoro di programmazione}
% }

\newglossaryentry{umlg} {
    name=\glslink{uml}{UML},
    text=UML,
    sort=uml,
    description={in ingegneria del software \emph{UML, Unified Modeling Language} (ing. linguaggio di modellazione unificato) è un linguaggio di modellazione e specifica basato sul paradigma object-oriented. L'\emph{UML} svolge un'importantissima funzione di ``lingua franca'' nella comunità della progettazione e programmazione a oggetti. Gran parte della letteratura di settore usa tale linguaggio per descrivere soluzioni analitiche e progettuali in modo sintetico e comprensibile a un vasto pubblico}
}
\newglossaryentry{WebApp} {
    name=\glslink{WebApp}{Web Application},
    text=WebApp,
    sort=WebApp,
    description={
        Una \emph{Web Application} (ing. applicazione web) è un'applicazione fruibile via web, ovvero un programma accessibile via web attraverso un network, tipicamente internet, sfruttando un'architettura tipica client-server. Le \emph{Web Application} sono applicazioni distribuite che funzionano su più dispositivi o ambienti, e sono spesso scritte in linguaggi di programmazione interpretati come \emph{PHP} o \emph{Python}. Le \emph{Web Application} sono molto popolari a causa della diffusione dei browser web e della possibilità di utilizzare un \emph{server web} per distribuire un'applicazione a più utenti
    }
}
\newglossaryentry{PoG} {
    name=\glslink{PoC}{Proof of Concept},
    text=PoC,
    sort=PoC,
    description={
        \emph{Proof of Concept} (ing. prova di fattibilità) è un termine usato in ingegneria del software e in sviluppo di sistemi per indicare un esempio o una realizzazione incompleta di un determinato metodo o idea per dimostrarne la fattibilità, in particolare per dimostrare la sua validità pratica. Un \emph{PoC} è spesso usato come dimostrazione iniziale che il progetto è fattibile e funzionante. Il \emph{PoC} è anche usato per raccogliere feedback che possono aiutare a migliorare il progetto
        }
    }


\makeglossaries

\bibliography{appendix/bibliography}

\defbibheading{bibliography} {
    \cleardoublepage
    \phantomsection
    \addcontentsline{toc}{chapter}{\bibname}
    \chapter*{\bibname\markboth{\bibname}{\bibname}}
}

% Spacing between entries
\setlength\bibitemsep{1.5\itemsep}

\DeclareBibliographyCategory{opere}
\DeclareBibliographyCategory{web}

\addtocategory{opere}{womak:lean-thinking}
\addtocategory{web}{site:agile-manifesto}

\defbibheading{opere}{\section*{Riferimenti bibliografici}}
\defbibheading{web}{\section*{Siti Web consultati}}


\captionsetup{
    tableposition=top,
    figureposition=bottom,
    font=small,
    format=hang,
    labelfont=bf
}

\hypersetup{
    %hyperfootnotes=false,
    %pdfpagelabels,
    colorlinks=true,
    linktocpage=true,
    pdfstartpage=1,
    pdfstartview=,
    breaklinks=true,
    pdfpagemode=UseNone,
    pageanchor=true,
    pdfpagemode=UseOutlines,
    plainpages=false,
    bookmarksnumbered,
    bookmarksopen=true,
    bookmarksopenlevel=1,
    hypertexnames=true,
    pdfhighlight=/O,
    %nesting=true,
    %frenchlinks,
    urlcolor=webbrown,
    linkcolor=RoyalBlue,
    citecolor=webgreen
    %pagecolor=RoyalBlue,
}

% Delete all links and show them in black
\if \isprintable 1
    \hypersetup{draft}
\fi

% Listings setup
\lstset{
    language=[LaTeX]Tex,%C++,
    keywordstyle=\color{RoyalBlue}, %\bfseries,
    basicstyle=\small\ttfamily,
    %identifierstyle=\color{NavyBlue},
    commentstyle=\color{Green}\ttfamily,
    stringstyle=\rmfamily,
    numbers=none, %left,%
    numberstyle=\scriptsize, %\tiny
    stepnumber=5,
    numbersep=8pt,
    showstringspaces=false,
    breaklines=true,
    frameround=ftff,
    frame=single
}

\definecolor{webgreen}{rgb}{0,.5,0}
\definecolor{webbrown}{rgb}{.6,0,0}

\newcommand{\sectionname}{sezione}
\addto\captionsitalian{\renewcommand{\figurename}{Figura}
                       \renewcommand{\tablename}{Tabella}}

\newcommand{\glsfirstoccur}{\ap{{[g]}}}

\newcommand{\intro}[1]{\emph{\textsf{#1}}}

% Risks environment
\newcounter{riskcounter}                % define a counter
\setcounter{riskcounter}{0}             % set the counter to some initial value

%%%% Parameters
% #1: Title
\newenvironment{risk}[1]{
    \refstepcounter{riskcounter}        % increment counter
    \par \noindent                      % start new paragraph
    \textbf{\arabic{riskcounter}. #1}   % display the title before the content of the environment is displayed
}{
    \par\medskip
}

\newcommand{\riskname}{Rischio}

\newcommand{\riskdescription}[1]{\textbf{\\Descrizione:} #1.}

\newcommand{\risksolution}[1]{\textbf{\\Soluzione:} #1.}

% Use case environment
\newcounter{usecasecounter}             % define a counter
\setcounter{usecasecounter}{0}          % set the counter to some initial value

%%%% Parameters
% #1: ID
% #2: Nome
\newenvironment{usecase}[2]{
    \renewcommand{\theusecasecounter}{\usecasename #1}  % this is where the display of
                                                        % the counter is overwritten/modified
    \refstepcounter{usecasecounter}             % increment counter
    \vspace{10pt}
    \par \noindent                              % start new paragraph
    {\large \textbf{\usecasename #1: #2}}       % display the title before the
                                                % content of the environment is displayed
    \medskip
}{
    \medskip
}

\newcommand{\usecasename}{UC}

\newcommand{\usecaseactors}[1]{\textbf{\\Attori Principali:} #1. \vspace{4pt}}
\newcommand{\usecasepre}[1]{\textbf{\\Precondizioni:} #1. \vspace{4pt}}
\newcommand{\usecasedesc}[1]{\textbf{\\Descrizione:} #1. \vspace{4pt}}
\newcommand{\usecasepost}[1]{\textbf{\\Postcondizioni:} #1. \vspace{4pt}}
\newcommand{\usecasealt}[1]{\textbf{\\Scenario Alternativo:} #1. \vspace{4pt}}

% Namespace description environment
\newenvironment{namespacedesc}{
    \vspace{10pt}
    \par \noindent  % start new paragraph
    \begin{description}
}{
    \end{description}
    \medskip
}

\newcommand{\classdesc}[2]{\item[\textbf{#1:}] #2}
